%# -*- coding: utf-8-unix -*-
\chapter{涡激振动}
\label{chap:viv}

圆柱横向振动的运动方程为
\begin{equation}
  m\ddot{y}+c\dot{y}+ky=F
  \label{eq:transmotion}
\end{equation}
这里$m$是振动结构的总质量,$c$是结构阻尼,$k$是弹簧常数,$F$是横向流体力。
当物体的振动频率与泻涡频率同步(即与作用在物体上的诱导力频率同步)时,
流体力与结构响应可以近似为
\begin{eqnarray}
  &&F(t)=F_0\sin(\omega t+\phi) \label{eq:Ft}\\
  &&y(t)=y_0\sin(\omega t) \label{eq:yt}
\end{eqnarray}
其中$\omega=2\pi f$,$f$是物体振动频率,$\phi$是流体力相位超前结构振动的值。

涡激振动问题中,我们感兴趣的物理量有泻涡频率$f$以及结构振幅$y_0$。
由定性分析可知,影响$f$和$y_0$的主要参数有结构物质量$m$、结构阻尼$c$、弹簧常数$k$、
圆柱直径$D$、流体密度$\rho$、运动粘度$\nu$、来流速度$U$、作用在结构上与来流垂直的
横向流体力$F_Y$和沿来流方向流体力$F_X$。这11个物理量共包含三个基本单位:
质量单位$[M]$,长度单位$[L]$和时间单位$[T]$。根据$\Pi$定理,相互关联的无因次化的
物理量有八个,?推荐的无因次化的物理量如表?所示。
这里$m_A=C_A m_d$是附加质量,$m_d$是圆柱的排水体积,$L$是圆柱长度。
$C_A\approx 1$是圆柱在静水中横向运动的附加质量系数。$f_N$是系统在水中的自然频率
$2\pi f_N=\sqrt{k/(m+m_A)}$。

将式\eqref{eq:Ft}和式\eqref{eq:yt}代入\eqref{eq:transmotion}中,并令等式两端
正弦项$\sin\omega t$和余弦项$\cos\omega t$前的系数分别相等,可得
\begin{eqnarray*}
  &&(k-m\omega^2)y_0=F_0\cos\phi \\
  &&c \omega y_0=F_0\sin\phi
\end{eqnarray*}
将上式中的物理量用表?中相应的无因次量表示,可得响应幅值$A^*$和响应频率$f^*$
\begin{eqnarray}
  &&{{A}^{*}}=\frac{1}{4{{\pi }^{3}}}\frac{{{C}_{Y}}\sin \phi }{({{m}^{*}}+{{C}_{A}})\zeta }{{\left( \frac{{{U}^{*}}}{{{f}^{*}}} \right)}^{2}}{{f}^{*}}
  \label{eq:A}\\
  &&{{f}^{*}}=\sqrt{\frac{{{m}^{*}}+{{C}_{A}}}{{{m}^{*}}+{{C}_{EA}}}}
  \label{eq:f}
\end{eqnarray}
其中$C_{EA}$是包括了总横向流体力中与物体加速度同相位的部分产生的``实效''附加质量,
由下式给出
\begin{equation}
  \[{{C}_{EA}}=\frac{1}{2{{\pi }^{3}}}\frac{{{C}_{Y}}\cos \phi }{{{A}^{*}}}{{\left( \frac{{{U}^{*}}}{{{f}^{*}}} \right)}^{2}}\]
  \label{eq:effaddedmass}
\end{equation}

\section{最大响应振幅与系统质量和阻尼的关系}
\label{sec:maxamp}

最大响应振幅是系统质量和阻尼怎样的函数是圆柱横向涡激振动中最基本的问题。
在文献中通常将该信息表示成$A^*_{\max}$关于一个参数,如$S_G$,的图像,
该参数与质量和阻尼的乘积成正比。[Griffin et al. (1975)]首先对已有的数据进行
综合汇编,形成``Griffin图'',这幅经典的图被[Skop \& Balas (1997)]更新。
选择质量和阻尼的乘积作为参数的逻辑来自对式\eqref{eq:A}的观察,并在许多文献中讨论过,
如[Bearman (1984)]。许多质量阻尼参数被提出并被普遍使用,如[Vickery \& Watkins (1964)]
作了最大振幅关于``稳定性参数''$K_s=\pi^2(m^*\zeta)$的图,[Scruton (1965)]将参数
取为Scruton数$Sc=(\pi/2)(m^*\zeta)$,而``Griffin图''所用的参数为
$S_G=2\pi^2S^2(m^*\zeta)$。

尽管``Griffin图''被实践工程师广泛使用,关于它有效性的问题被[Sarpkaya (1978,1979)]
和[Bearman (1984)]明确指出。[Bearman (1984)]认为,在低质量比$m^*$的情形下,
即使$m^*\zeta$或等价的$(m^*+C_A)\zeta$固定,频率比$f^*$仍受$m^*$的单独影响,
这从式\eqref{eq:f}可以明显看出。因此,可以预期最大振幅$A^*$并不只是组合参数
$m^*\zeta$的函数。然而,$m^*$的单独影响在一些条件下可能是小到忽略不计的。
这就提出了一个重要问题:当$m^*$和$\zeta$处于什么范围内,才能获得$A^*_{\max}$关于
$m^*\zeta$的较精确的关系。

[Sarpkaya (1978,1979)]认为只有当$S_G>1.0$时才可使用组合参数$S_G$。
[Sarpkaya (1993,1995)]进一步阐述,动力响应受到$m^*$和$\zeta$的单独影响,而不只是
$m^*\zeta$。这一观点受到了[Zdravkovich (1990)]的支持,他认为在海洋工程的应用中,
质量比和阻尼系数必须被看作单独的参数。在另一方面,[Ramberg \& Griffin (1981)]
的试验结果表明,即使在相当低的质量比时,最大峰值也与组合参数吻合良好,并且$S_G$比
[Sarpkaya (1978,1979)]接受的范围$S_G>1.0$要小。
