%# -*- coding: utf-8-unix -*-
%%==================================================
%% abstract.tex for SJTU Master Thesis
%%==================================================

\begin{abstract}
很久以前人们就观察到很多船的船行波被限制在与航迹夹角
$-\psi_{\max}\le\psi\le\psi_{\max}$的锥形范围内,其中锥形半角$\psi_{\max}$
明显小于开尔文角$\psi^K\approx19^\circ28'$。很多理论已被提出以解释这些现象。
基于波浪干涉效应的两点兴波模型表明,在无限水深的静水中高速航行的单体船首尾散波间的纵向干涉或双体船
两片体首波间的横向干涉会在开尔文楔内显著小于开尔文角的射线$\psi^K$处产生波高最大的波。
本文考虑波浪干涉效应对船长$L$的单体船或船长$L$间距$S$且具有相同片体的双体船在无限水深的静水中以航速$V$航行时远场波的影响,
通过在船体表面分布的点源来代表船体。
研究报告了七艘主尺度范围广泛(长宽比、船长吃水比、型宽吃水比、半进流角)的简单船体在很大范围的弗洛德数$F\equiv V/\sqrt{gL}$
和无因次片体间距$s\equiv S/L$情况下的系统计算结果。这里,$g$为重力加速度。
单体船或双体船产生的波高最大的波所在的角度,即主要兴波角,由一种简单实用的方法获得。
这种方法基于数值确定由Hogner近似和驻相法估计的傅里叶科钦表达式中波幅函数的峰值。
本文主要的一般性结论是,尽管船行波幅值的大小受船体形状的强烈影响,但由于干涉效应造成的主要兴波角的大小受船体形状的影响很弱。
因此,与最大波高相关联的主要兴波角的大小主要是船行波的运动学特性。该发现的重要结果是,对于一般性的单体船和双体船,
主要兴波角的大小可由弗洛德数$F$和/或片体间距$s$通过简单的解析关系式得出。这些解析关系式在文中通过系统的数值计算得出。
单体船主要兴波角的解析式考虑了船首尾区域散波的纵向干涉效应和左右舷散波的横向干涉,而双体船主要兴波角的解析式还考虑了两片体产生的散波间的干涉,
因此相比两点兴波模型的初步分析给出的$\psi_{\max}$的估计更加精确。

\keywords{远场波、波浪干涉、主要兴波角、高速船}
\end{abstract}

\begin{englishabstract}
It has long been observerd that the far-field waves created by a ship that steadily
advance at high speed in calm water of large depth appear to be contained within
a wedge $-\psi_{\max}\le\psi\le\psi_{\max}$ where $\psi_{\max}$ can be significantly
smaller than the cusp angle $\psi^K\approx19^\circ28'$. Several alternative
theories have been put forward to explain this phenomenon. The two-point wavemaker
model based on wave-interference effects shows that the longitudinal interference
of waves created by the bow and the stern of a monohull ship or the lateral 
interference of waves created by the twin hulls of a catamaran can lead to largest
waves at angles that are significantly shaller that $\psi^K$. 
In this paper, the wave-interference effects on the far-field waves created by
a monohull ship of length $L$ or by a catamaran with identical twin hulls of
length $L$ at a lateral separation distance $S$, that advances at constant speed
$V$ along a straight path in calm water of large depth are considered. The hulls
are represented via a continuous distribution of sources around hull surface.
Systematic computations are performed for a wide range of Froude numbers 
$F\equiv V/\sqrt{gL}$, hull spacings $s\equiv S/L$, and seven simple 
mathematically-defined hulls that correspond to a broad range of main hull-shape
parameters (beam/length, draft/length, beam/draft, waterline entrance angle). Here,
$g$ denotes the acceleration of gravity. The dominant ray angles, where the largest
waves created by the monohull ships or the catamarans are found, are determined 
numerically via a realistic yet practical method. This method is based on the 
numerical determination of the peaks of the amplitude function in the Fourier-Kochin
representation of far-field ship waves, evaluated by means of the Hogner 
approximation, and the stationary-phase approximation. 
The main general conclusion of the study is that, although the amplitudes of 
the waves created by a ship are strongly influenced by the hull geometry 
(as well known), the dominant ray angles associated with the largest waves created 
by a ship (due to interference effects) only weakly depend on the hull geometry. 
Thus, the ray angles associated with the dominant waves are mostly a kinematic 
feature of the waves created by a ship. An important practical consequence of this 
finding is that the wake angles that correspond to the dominant waves created by a 
ship can be explicitly determined in terms of the Froude number $F$ for monohull 
ships, or in terms of the Froude number $F_s$ and the hull spacing $s$ for 
catamarans, via simple analytical relations. 
These relations, obtained via systematic numerical computations, are given here.
The numerical determined dominant ray angles take into account 
the longitudinal interference effects of the waves created by the bow and the stern
as well as the lateral interfence effects of the waves created by the port and 
starboard side of the monohull ship and by the twin hulls of the catamran. Thus,
the analytical relations obtained here provide more accurate predictions of
the dominant ray angles $\psi_{\max}$ of high-speed monohull ships and catamarans
than that derived by the simple two-point wavemaker model.

\englishkeywords{farfield waves, interference effects, dominant ray angles, 
high-speed vessels}
\end{englishabstract}

