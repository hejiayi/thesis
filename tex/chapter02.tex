%# -*- coding: utf-8-unix -*-
%%==================================================
%% chapter02.tex for SJTU Master Thesis
%% Encoding: UTF-8
%%==================================================

\chapter{船波基本理论}
\label{chap:shipwav}

\section{船舶兴波问题的数学提法}
\label{sec:mathsform}

考虑船长为$L$的船以航速$V$在无限深广的静水中直线航行时的兴波。假定流体为均质、
不可压、无粘性的理想流体;假定流动有势无旋;假定船舶兴波为微幅波,即波幅与波长
相比为小量;忽略表面张力的影响。观察流动的参考系是固结在船体上与船体一起运动的
伽利略参考系$\mathbf{X}=(X,Y,Z)$,时间为$T$。流场速度势$\bar{\Phi}(\mathbf{X},T)$可
看作均匀来流的速度势$(-VX,0,0)$与扰动速度势$\Phi(\mathbf{X},T)$的叠加,
将未扰动的自由面定义为$Z=0$且$Z$轴竖直向上,$X$轴与船的航向一致且指向船首。

考察自由面情况。设自由面方程为$Z=E(X,Y,T)$,则运动学条件为
\begin{equation*}
  E_X\bar{\Phi}_X+E_Y\bar{\Phi}_Y+E_T-\bar{\Phi}_Z=0
\end{equation*}
将$\bar{\Phi}=-VX+\Phi$代入上式得
\begin{equation*}
  VE_X=E_T+E_X\Phi_X+E_Y\Phi_Y-\Phi_Z
\end{equation*}
忽略二阶小量$E_X\Phi_X$和$E_Y\Phi_Y$得
\begin{equation}
  VE_X=E_T-\Phi_Z
  \label{eq:fskinem}
\end{equation}
式\eqref{eq:fskinem}是线性化的非定常运动自由面运动学条件。
对波面以及远前方自由面应用非定常伯努利方程,有
\begin{equation*}
  \frac{\partial\bar{\Phi}}{\partial T}+\frac{P}{\rho}+\frac{1}{2}(\bar{\Phi}_X^2
  +\bar{\Phi}_Y^2+\bar{\Phi}_Z^2)+gE
  =\frac{P_a}{\rho}+\frac{1}{2}V^2
\end{equation*}
将$\bar{\Phi}_X=-V+\Phi_X$代入上式,并利用等压条件,可得
\begin{equation*}
  gE=V\Phi_X-\frac{1}{2}(\Phi_X^2+\Phi_Y^2+\Phi_Z^2)-\Phi_T
\end{equation*}
略去二阶小量$(\Phi_X^2+\Phi_Y^2+\Phi_Z^2)/2$,可得
\begin{equation}
  gE=V\Phi_X-\Phi_T
  \label{eq:fsdynam}
\end{equation}
式\eqref{eq:fsdynam}是线性化的非定常运动自由面动力学条件。
联立\eqref{eq:fskinem}和\eqref{eq:fsdynam}消去波高$E$,可得
\begin{equation}
  [g\partial_Z+(V\partial_X-\partial_T)^2]\Phi=0
  \label{eq:fscombin}
\end{equation}
式\eqref{eq:fscombin}是非定常运动线性自由面条件。以$U$,$U/g$,$U^2/g$,
$U^3/g$为特征速度、时间、长度、速度势,定义无因次化的物理量
\begin{equation*}
  t=Tg/U,\quad\mathbf{x}=\mathbf{X}g/U^2,\quad\phi=\Phi g/U^3
\end{equation*}
作为无因次化的时间、坐标、速度势。基于这些无因次化量的线性自由面条件为
%
%%用船长$L$以及航速$V$定义无因次化的坐标系
%%$\mathbf{x}\equiv\mathbf{X}/L}$、扰动速度势$\phi=\Phi/VL$、速度场
%%$\nabla_\mathbf{x}\equiv(\phi_x,\phi_y,\phi_z)\equiv(\Phi_X,\Phi_Y,\Phi_Z)/V$
%%以及时间$t=TV/L$。则式\eqref{eq:fscombin}变为无因次形式
\begin{equation}
  [\partial_z+(\partial_x-\partial_t)^2]\phi=0
  \label{eq:fsnondim}
\end{equation}

在本论文中,我们关心稳定后的流场。众所周知,为了解的确定性,定常的自由面重力流
需要引入辐射条件。替代辐射条件的一个简单方法是将定常流作为辅助参数$\epsilon$
趋向零时随时间而变的流场的极限。时变流场的速度势为$\varphi(\mathbf{x},t)=\phi(\mathbf{x})\exp(\epsilon t)$。于是将原问题转化为传统的初边值问题(初始条件为当$t\to-\infty$时,$\varphi=0$,$\varphi_t=0$)。与空间速度势$\phi(\mathbf{x})$对应的自由面条件为
\begin{equation*}
  [\partial_z+(\partial_x-\epsilon)^2]\phi=0
\end{equation*}
或等价形式
\begin{equation*}
  \phi_z+\phi_{xx}-2\epsilon\phi_x+\epsilon^2\phi=0
\end{equation*}
当忽略上式中的项$\epsilon^2\phi$时---这对于$\epsilon\ll1$来说是合理的---自由面条件
与通常使用的Rayleigh人工粘性得出的自由面条件相同。

则无因次化形式船舶兴波的边值问题为
%%
\begin{equation}\label{eq:govrn}
\left\{ 
  \begin{array}{l}
  \nabla^2_{\mathbf{x}}\phi\equiv\phi_{xx}+\phi_{yy}+\phi_{zz}=0 \\
  F^2\phi_{xx}+\phi_z=0,\quad\mathbf{x}\in\Sigma^F\\
  \nabla\phi\cdot\mathbf{n}=n_x, \quad \mathbf{x}\in\Sigma^H\\
  \phi_z\rightarrow0, \quad z\rightarrow -\infty\\
  \phi\rightarrow0, \quad h\equiv\sqrt{x^2+y^2}\rightarrow\infty
\end{array} \right.
\end{equation}
%%
这里,$\Sigma^H$代表船体平均湿表面积,$\Sigma^F$代表位于船体表面$\Sigma^H$之外
的平均(未扰动)自由面$z=0$。求解边值问题\eqref{eq:govrn}可得船舶静水兴波速度势,
解决该问题目前常用格林函数法,该方法将在下一章介绍。

\section{平面前进波}
\label{sec:freewav}

船静水兴波速度势可分为局部流动部分 (local flow component) 和波动部分 (wave 
component)。其中局部流动部分随离开船体距离的增加迅速衰减,而波动部分衰减较慢。
令$\mathbf{u}^L$和$\mathbf{u}^W$分别是与局部流动和波动部分对应的流速,则
\begin{equation}
\mathbf{u}^W\sim 1/\sqrt{h},\,\mathbf{u}^L\sim 1/h^3, \quad\text{当}\,h\rightarrow\infty
  \label{eq:uWuL}
\end{equation}
因此,在离开船体的一小段距离之后局部流动速度$\mathbf{u}^L$相比波动部分$\mathbf{u}^W$
可忽略不计,离开船体的外传波除了波幅受船体影响之外,几乎不受船体的影响。
这些离开船体一段距离的波称为自由波。船体表面条件决定了波幅的大小,但是除了在船体
附近很小的区域内,不影响自由波的传播。

自由波满足的控制方程和边界条件为
\begin{equation}\label{eq:govrnfreewav}
\left\{ 
\begin{array}{l}
  \phi_{xx}+\phi_{yy}+\phi_{zz}=0,\quad z\le0 \\
  F^2\phi_{xx}+\phi_z=0,\quad z=0\\
  \phi_z\rightarrow0, \quad z\rightarrow -\infty\\
  \phi\rightarrow0, \quad h\equiv\sqrt{x^2+y^2}\rightarrow\infty
\end{array} \right.
\end{equation}

可以验证,如下基本初等波
\begin{equation}
  W(\mathbf{x})=\mathrm{e}^{kz+\mathrm{i}(\alpha x+\beta y)}
  \label{eq:elewav}
\end{equation}
满足\eqref{eq:govrnfreewav}中的控制方程、海底边界条件。这里,
\begin{equation}
  k=\sqrt{\alpha^2+\beta^2}
  \label{eq:wavnumber}
\end{equation}
当满足
\begin{equation}
  \Delta(\alpha,\beta;F)\equiv F^2\alpha^2-k=0
  \label{eq:disper}
\end{equation}
\eqref{eq:elewav}还满足自由面边界条件。
方程$\Delta=0$称为色散关系,函数$\Delta(\alpha,\beta;F)$是色散函数。色散关系
在$(\alpha,\beta)$平面内形成的曲线称为色散曲线。

为下文的叙述做准备,定义如下符号
\begin{eqnarray}
  &\mathbf{k}\equiv(\alpha,\beta)\equiv k(\cos\gamma,\sin\gamma),\quad \nabla_{\matbf{k}}\equiv\left(\frac{\partial}{\partial\alpha},\frac{\partial}{\partial\beta}\right) \label{eq:kbfdef} \\
  &\mathbf{h}\equiv(x,y)\equiv h(\cos\psi,\sin\psi),\quad
  \nabla_{\mathbf{h}}\equiv\left(\frac{\partial}{\partial x},\frac{\partial}{\partial y}\right) \label{eq:hbfdef}\\
  &\theta\equiv\alpha x+\beta y\equiv\mathbf{k}\cdot\mathbf{h}\equiv kh\cos(\gamma-\psi) \label{eq:thetadef}
\end{eqnarray}
$k$,$\alpha$和$\beta$是无因次化的波数。$k$,$\alpha$,$\beta$以及无因次化的波长
$\lambda\equiv 2\pi/k$与有量纲的波数$K$, $K^x$, $K^y$以及波长$\Lambda$的关系为
\begin{equation}
  (k,\alpha,\beta)\equiv(K,K^x,K^y)L,\quad \lambda=\Lambda/L
  \label{eq:klambdim}
\end{equation}

色散关系\eqref{eq:disper}定义了两条色散曲线,这两条曲线关于$\alpha=0$以及$\beta=0$
对称。色散关系\eqref{eq:disper}可以表示成如下等价形式
\begin{equation}
  \alpha^2_F=k_F,\quad\text{这里}\quad k_F\equiv\sqrt{\alpha^2_F+\beta^2_F}
  ,\,(\alpha_F,\beta_F,k_F)\equiv F^2(\alpha,\beta,k)
  \label{eq:disper1}
\end{equation}
或参数方程形式
\begin{equation}
  \begin{array}{l}
    \alpha_F=\pm\sqrt{1+q^2},\,\beta_F=q\sqrt{1+q^2}\\
    k_F=1+q^2,\quad\text{这里}\quad -\infty<q<\infty
  \end{array}
  \label{eq:disper2}
\end{equation}
$\bullet$插入色散曲线。

由色散关系\eqref{eq:disper}得,波长$\lambda\equiv 2\pi/k$为
\begin{equation}
  \lambda=2\pi F^2\cos^2\gamma\le2\pi F^2\equiv \lambda^{\max}
  \label{eq:wavlendef}
\end{equation}

\eqref{eq:wavlendef}表明,航速为$V$的船产生的波长$\Lambda$的范围为
\begin{equation}
  \Lambda\le\Lambda^{\max}\equiv2\pi V^2/g
  \label{eq:Wavlen}
\end{equation}
波长最大的波浪传播方向为船的航向。
因此,当弗洛德数$F\equiv V/\sqrt{gL}$满足
\begin{equation}
  F>1/\sqrt{2\pi}\approx0.4
  \label{eq:hullspd}
\end{equation}
时,最大波长$\Lambda^{\max}$超过船长$L$。

波峰线是相位$\theta$的等值线。波峰的前进方向为
\begin{equation}
  \nabla_{\mathbf{h}}\theta\equiv(\partial\theta/\partial x,\partial\theta/\partial y)=(\alpha,\beta)\equiv k(\cos\gamma,\sin\gamma)\equiv\mathbf{k}
  \label{eq:dtheta}
\end{equation}
波峰前进的速度,即相速度$\mathbf{v}_p$为
\begin{equation}
  \mathbf{v}_p\equiv(v_p^x,v_p^y)=v_p(\cos\gamma,\sin\gamma)\equiv 
  v_p\frac{\mathbf{k}}{k}
  \label{eq:vp}
\end{equation}
由于对应每一波峰线的相位总是等于一个固定常数,则
\begin{equation}
  \frac{\mathrm{d}\theta}{\mathrm{d}t}=\alpha\frac{\mathrm{d}x}{\mathrm{d}t}
  +\beta\frac{\mathrm{d}y}{\mathrm{d}t}=\alpha v_p^x+\beta v_p^y=kv_p=0
  \label{eq:dthetadt}
\end{equation}
由此可得$v_p=0$。这表明从固结在船体上的伽利略坐标系观察船的静水兴波,波形是定常的。

取空间固定坐标系$(x',y',z')$,并假设$t=0$时两坐标系重合。从而此二坐标系有如下
关系
\begin{equation}
  x'=x+t;\,y'=y;z'=z
  \label{eq:xtrans}
\end{equation}
则空间固定坐标系中的相位$\theta'$为
\begin{equation}
  \theta'=\alpha(x'-t)+\beta y'
  \label{eq:thetafix}
\end{equation}
由$\mathrm{d}\theta'/\mathrm{d}t=0$可得空间固定坐标系中的相速度$v'_p$
\begin{equation}
  v'_p=\frac{\alpha}{k}=\cos\gamma
  \label{eq:vpfix}
\end{equation}

波群速度为
\begin{equation}
  \mathbf{v_g}\equiv(v_g^x,v_g^y)=\frac{1}{2}(\cos^2\gamma,\sin\gamma\cos\gamma)-(1,0)
  \label{eq:vg}
\end{equation}
\eqref{eq:vg}中的$(1,0)$来源于我们将坐标系取为固结在船体上跟船一起运动的
伽利略坐标系。
我们有
\begin{equation}
  \left\{
    \begin{array}{r}
      -v_g^x \\
      v_g^y
    \end{array}
  \right\}
  =\frac{1}{2}\left\{
    \begin{array}{l}
    1+\sin^2\gamma\\
    \sin\gamma\cos\gamma
  \end{array}
\right\}
\quad\text{这里}\quad v_g=\frac{1}{2}\sqrt{1+3\sin^2\gamma}
\label{eq:vg2}
\end{equation}


波群速度在大地坐标系下为
\begin{equation}
  \mathbf{v'_g}\equiv(v_g^x+1,v_g^y)
  \label{eq:vgtrans}
\end{equation}

由\eqref{eq:vg}和\eqref{eq:vgtrans}得,
\begin{equation}
  \mathbf{v'_g}\equiv v'_g(\cos\gamma,\sin\gamma)
  \label{eq:vgfixed}
\end{equation}
由\eqref{eq:vpfix}可知,这里$v'_g=v'_p/2$。
表明在空间固定坐标系中观察,波群速度$\mathbf{v'_g}$与相速度$\mathrm{v'_p}$
共线,且在数值上等于相速度的一半。但是在随体坐标系中,两者并不共线。






\section{初始条件和远场条件}
\label{sec:initfar}

满足色散关系\eqref{eq:disper}的波函数\eqref{eq:elewav}是满足
\eqref{eq:govrnfreewav}中的Laplace方程、自由面边界条件、海底条件的基本解。
因此这些基本解在由\eqref{eq:disper}定义的色散曲线上的叠加也满足Laplace
方程和这些边界条件,即
\begin{equation}
  \phi^W(\mathbf{x})=\sum_{\Delta=0}\int_{\Delta=0}\mathrm{d}s\,a^\phi
  W(\mathbf{x},\alpha,\beta)\quad\text{这里}\quad (\alpha,\beta)\in\Delta=0
  \label{eq:wavsum}
\end{equation}
这里,求和$\Sigma$遍历所有的色散曲线,点$(\alpha,\beta)$是色散曲线上的点,
$W(\mathbf{x},\alpha,\beta)$是基本波函数\eqref{eq:elewav}。
这里$\mathrm{d}s$是色散曲线上的弧长微元,$a^\phi$是一般化的(未知的)波幅函数。
\eqref{eq:wavsum}假定的色散曲线是通过以弧长$s$为参数的参数方程定义的。
然而,其他用来表示色散曲线的参数也可以使用。

因此,离开船体一段距离的自由波及其速度势$\phi^W(\mathbf{x})$可以表示为基本波函数
在色散曲线\eqref{eq:disper}上的一维Fourier叠加。然而,由\eqref{eq:elewav}定义
的波函数$W$不满足初始条件。因此,Fourier叠加\eqref{eq:wavsum}无法令人满意地代表
自由波。比如,这种表示方式无法排除向船前方传播的波。正确的表示方式需要考虑初始条件。
下面将给出满足初始条件和\eqref{eq:govrnfreewav}中的远场条件的基本波。

满足初始条件和远场条件的Fourier叠加形式为
\begin{equation}
  \phi^W(\mathbf{x})=\sum_{\Delta=0}\int_{\Delta=0}\mathrm{d}s\,a^\phi
  H(\sigma^\Delta,\sigma^\theta)
  W(\mathbf{x},\alpha,\beta)\quad\text{这里}\quad (\alpha,\beta)\in\Delta=0
  \label{eq:wavsumm2}
\end{equation}
这里$\sigma^\Delta$和$\sigma^\theta$为
\begin{eqnarray}
  &\sigma^\Delta\equiv-\mathrm{sign}(\alpha)\label{eq:sigd}\\
  &\sigma^\theta\equiv\mathrm{sign}(\theta)\equiv\mathrm{\cos(\gamma-\psi)}
  \equiv\mathrm{sign}(\alpha x+\beta y) \label{eq:sigtheta}
\end{eqnarray}
$H(\cdot)$为单位阶跃函数。由\eqref{eq:wavsumm2},\eqref{eq:sigd}和\eqref{eq:sigtheta}
得
\begin{equation}
  \left\{
    \begin{array}{l}
      \psi-\pi/2<\gamma<\psi+\pi/2\\
      \psi+\pi/2<\gamma<\psi+3\pi/2
    \end{array}
  \right\}
  \quad\text{如果}\quad
  \left\{
    \begin{array}{l}
      0<\sigma^\Delta\\
      \sigma^\Delta<0
    \end{array}
  \right\}
  \label{eq:allowed}
\end{equation}
\eqref{eq:allowed}定义了色散曲线中允许被叠加的部分。这个被叠加的部分是定义在傅里叶
平面$(\alpha,\beta)$中,依赖于物理平面$(x,y)$中角度$\psi$。

在按\eqref{eq:allowed}方式限制的色散曲线$\Delta=0$上进行Fourier叠加的
表达式\eqref{eq:wavsum}与表达式\eqref{eq:wavsumm2}是等价的。表达式\eqref{eq:wavsumm2}
满足\eqref{eq:govrnfreewav}中的Laplace方程、远场边界条件、自由面边界条件和海底
边界条件,也满足初始条件。因此,表达式\eqref{eq:wavsumm2}可以代表由船体产生的
离开船体一小段距离之后传播的自由波。式\eqref{eq:wavsumm2}中一般化的波幅函数只能
通过船体表面边界条件确定。

\section{远场波}
\label{sec:farfieldiwav}

在远场$h\gg1$,自由波表达式\eqref{eq:wavsumm2}可以进行进行简化。考虑$h\gg1$可以
将傅里叶平面$(\alpha,\beta)$中由色散关系定义的色散曲线上的点与物理空间中的远场波
联系起来。远场波形可以通过色散函数由简单的解析式确定。

考虑自由波\eqref{eq:wavsumm2}在远场$h\gg 1$的情况。Fourier表达式\eqref{eq:wavsumm2}
表示成
\begin{equation}
  \phi^W(\mathbf{x})=\sum_{\Delta=0}\int_{\Delta=0}\mathrm{d}s\,\hat{a}\mathrm{e}
  ^{\mathrm{i}h\varphi}
  \label{eq:wavsum3}
\end{equation}
这里$h\equiv\sqrt{\xi^2+\eta^2}$,其中$(\xi,\eta)$是以$V^2/g$无因次化的坐标系。
$\hat{a}\equiv\hat{a}(s)$和$\varphi=\varphi(s)$是色散曲线$\Delta=0$的弧长$s$
的函数,定义为
\begin{eqnarray}
  &\hat{a}\equiv H(\sigma^\Delta\sigma^\theta)a^\phi\mathrm{e}^{kz}
  \label{eq:aphiaz}\\
  &\varphi\equiv\frac{\theta}{h}\equiv\alpha\cos\psi+\beta\sin\psi\equiv k\cos(\gamma-\psi)
  \label{eq:phidef}
\end{eqnarray}
这里,$\theta$是由\eqref{eq:thetadef}定义的相位函数。
积分\eqref{eq:wavsum3}中的三角函数$\mathrm{e}^{\mathrm{i}h\varphi}}$是随$s$
振动的函数。振动周期由关系式$h\varphi_s\delta s=2\pi$确定,即
\begin{equation}
  \delta s=\frac{2\pi}{h\varphi_s}
  \label{eq:osciperiod}
\end{equation}
其中$\varphi_s=\mathrm{d}\varphi/\mathrm{d}s$是相位函数$\varphi(s)$在色散曲线
$\Delta=0$上点$s$处的导数。因此,振动周期$\delta s$是弧长$s$的函数。
式\eqref{eq:osciperiod}表明随着$h$的增加,局部振动周期$\delta s$变小,三角函数
$\mathrm{e}^{\mathrm{i}h\varphi}$振动更加快速。因此三角函数$\mathrm{e}^{\mathrm{i}h\varphi}$在远场$h\gg1$是快速振动的函数,除了在满足$\varphi_s(s)=0$的点$s$处,这样的
点称为驻相点。在远场$h\gg1$,三角函数$\mathrm{e}^{\mathrm{i}h\varphi}$快速振动的
部分对积分\eqref{eq:wavsum3}的贡献相互抵消,因此对式\eqref{eq:wavsum3}的主要贡献
来自驻相点$\varphi_z=0$。

由色散关系\eqref{eq:disper}、相位函数\eqref{eq:phidef}和$\varphi_s=0$得
\begin{equation}
  \frac{\mathrm{d}}{\mathrm{d}\gamma}\left(
  \frac{\cos\gamma\cos\psi+\sin\gamma\sin\psi}{F^2\cos^2\gamma}\right)=0
  \label{eq:phiz=0}
\end{equation}
由此可得
\begin{equation}
  \tan\psi*\equiv\frac{y}{x}=\frac{\tan\gamma}{1+2\tan^2\gamma}
  \label{eq:statfaz1}
\end{equation}

%
% 
由\eqref{eq:disper2}和\eqref{eq:phidef}得,相位函数$\varphi$、导函数
$\varphi'\equiv d\varphi/dq$和$\varphi''\equiv d^2\varphi/dq^2$为
%
\begin{subequations}\label{eq:faz}\begin{eqnarray}
&&\varphi\equiv(q\sin\hspace{-0.1em}\psi-\cos\hspace{-0.1em}\psi)
\sqrt{1\hspace{-0.1em}+q^2}\label{eq:faz-a}\\
&&\varphi'\equiv\frac{(1\hspace{-0.1em}+2\hspace{0.05em}q^2)\sin\hspace{-0.1em}\psi
-q\cos\hspace{-0.1em}\psi}{\sqrt{1\hspace{-0.1em}+q^2}}\label{eq:faz-b}\\
&&\varphi''\equiv\frac{q\hspace{0.1em}
(3+2\hspace{0.05em}q^2)\sin\hspace{-0.1em}\psi
-\cos\hspace{-0.1em}\psi}{(1\hspace{-0.1em}+q^2)^{3/2}}\label{eq:faz-c}
\end{eqnarray}\end{subequations}
%
这里$\psi$重定义为
%
\begin{equation}\label{eq:psidef}
  \tan\hspace{-0.1em}\psi\equiv\frac{y}{-x}
\end{equation}
%

驻相点由$\varphi'=0\hspace{0.05em},$确定,即
%
\begin{equation}\label{eq:statfaz}
(1\hspace{-0.1em}+2\hspace{0.1em}q^2)\tan\hspace{-0.1em}\psi-q=0
\end{equation}
%
在开尔文锥形内,即$-\psi_K<\psi<\psi_K$,这里$\psi^K\approx19^\circ28'$,
关系式\eqref{eq:statfaz}有两个不等的根$q^T$和$q^D$
%
 \begin{equation}\label{eq:qTD}
 q^T\equiv\frac{1\hspace{-0.1em}-\sqrt{1\hspace{-0.1em}-8\tan^2\!\psi}}
 {4\tan\hspace{-0.1em}\psi}
 \hspace{0.5em}\mbox{和}\hspace{0.5em}
 q^D\equiv\frac{1\hspace{-0.1em}+\sqrt{1\hspace{-0.1em}-8\tan^2\!\psi}}
 {4\tan\hspace{-0.1em}\psi}
 \end{equation}
 %
分别对应于横波和散波。
在波峰尖顶角 (cusp lines) $\psi=\pm\hspace{0.05em}\psi_K,$ $q^T$和$q^D$相互重合
%
\begin{equation}\label{eq:qcusp}
q^T\!=\pm\hspace{0.1em}q^C\hspace{0.5em}\mbox{和}\hspace{0.5em}
q^D\!=\pm\hspace{0.1em}q^C\hspace{0.5em}\mbox{这里}\hspace{0.5em}
q^C\hspace{-0.1em}\equiv1/\sqrt{2}\approx0.7
\end{equation}
%

由\eqref{eq:faz-a}、\eqref{eq:faz-c}和\eqref{eq:qTD}得
相位函数$\varphi$和二阶导数$\varphi''$在驻相点$q=q^D$处的值
$\varphi^D$和$\varphi''_D$为
%
\begin{subequations}\label{eq:fazD}\begin{eqnarray}
&&-\hspace{0.1em}\varphi^D\hspace{-0.1em}=\frac{3-\sqrt{1-8\tan^2\!\psi}}
{8\hspace{0.1em}\sqrt{2}}\,
\frac{\cos\hspace{-0.1em}\psi}{|\hspace{-0.1em}\tan\hspace{-0.1em}\psi\hspace{0.05em}|}
\,\Delta^\psi\label{eq:fazD-a}\\
&&\varphi''_D=2\hspace{0.1em}\sqrt{2}\,\sqrt{1\hspace{-0.1em}-8\tan^2\!\psi}\,
|\hspace{-0.1em}\sin\hspace{-0.1em}\psi\hspace{0.05em}|\hspace{0.1em}/\Delta^\psi
\label{eq:fazD-b}\\
&&\mbox{这里}\hspace{0.5em}\Delta^\psi\equiv
\sqrt{1\hspace{-0.1em}+4\tan^2\!\psi+\sqrt{1-8\tan^2\!\psi}}\label{eq:fazD-c}
\end{eqnarray}\end{subequations}
%
类似的, $\varphi$和$\varphi''$在$q=q^T$的值$\varphi^T$和$\varphi''_T$为 
%
\begin{subequations}\label{eq:fazT}\begin{eqnarray}
&&-\hspace{0.1em}\varphi^T\!=\frac{3+\sqrt{1-8\tan^2\!\psi}}
{2\hspace{0.1em}\sqrt{2}\hspace{0.2em}\Delta^\psi}\label{eq:fazT-a}\\
&&\varphi''_T=-\cos^2\!\psi\,
\sqrt{\frac{1\hspace{-0.1em}-8\tan^2\!\psi}{2}}\,\Delta^\psi\label{eq:fazT-b}
\end{eqnarray}\end{subequations}
%
式\eqref{eq:fazD-b}和\eqref{eq:fazT-b}表明$0\le\varphi''_D$和 
$\varphi''_T\le0\hspace{0.05em}.$ 在波峰尖顶角$\psi=\pm\psi_K,$ 
我们有$\tan^2\!\psi=1/8$,此时$\varphi''_D$和$\varphi''_T$为零。

在远场$1\ll h\hspace{0.05em},$ 积分\eqref{eq:wavsum3}可以通过驻相法得到解析近似
%
\begin{subequations}\label{eq:ZTD}\begin{eqnarray}
    &&\phi^W(\mathbf{x})\approx\sqrt{\frac{2\pi}{h}}
\,\hspace{0.05em}
(\hspace{0.05em}\phi^D\hspace{-0.1em}+\phi^T\hspace{0.05em})
\hspace{0.5em}\mbox{这里}\label{eq:ZTD-a}\\
&&\phi^D\equiv \frac{\hat{a}^D}
{\sqrt{\varphi''_D}}
\hspace{0.2em}e^{\hspace{0.1em}\mathrm{i}\hspace{0.1em}(h\hspace{0.1em}\varphi^D
+\hspace{0.1em}\pi/4)}\label{eq:ZTD-b}\\
&&\phi^T\equiv \frac{\hat{a}^T}
{\sqrt{-\varphi''_T}}
\hspace{0.2em}e^{\hspace{0.1em}\mathrm{i}\hspace{0.1em}(h\hspace{0.1em}\varphi^T
-\hspace{0.1em}\pi/4)}\label{eq:ZTD-c}
\end{eqnarray}\end{subequations}
%
这里$\hat{a}^D$和$\hat{a}^T$是由\eqref{eq:aphiaz}定义的一般化的波幅函数$\hat{a}$
在$q=q^D$或$q=q^T$处的值。渐近展开式\eqref{eq:ZTD}在波峰尖顶角附近无效,因为
二阶导数$\varphi''_D$和$\varphi''_T$在此处为零。

波峰线是相位的等值线,横波的波峰线是$h\varphi^T-\pi/4=\text{常数}$,散波的波峰
线是$h\varphi^D+\pi/4=\text{常数}$。令该常数为$-2n\pi$,则得如下参数方程
\begin{subequations}
  \label{eq:wavpat}
  \begin{eqnarray}
     \xi_\pm&=&(2n\pi\pm\pi/4)\cos\psi/\varphi_\pm\label{eq:wavpat-a}\\
    -\eta_\pm&=&(2n\pi\pm\pi/4)\sin\psi/\varphi_\pm\label{eq:wavpat-b}
  \end{eqnarray}
\end{subequations}
这里$\varphi_+\equiv\varphi^D$,$\varphi_-\equiv\varphi^T$。

$\bullet$插入开尔文波形图
